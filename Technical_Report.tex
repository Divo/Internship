\documentclass[a4paper,12pt, titlepage]{article}
%\usepackage{color}
%\definecolor{light-gray}{gray}{0.95}

\usepackage{xcolor}
\usepackage{alltt}
\usepackage{url}
\usepackage{tikz}
\usepackage{ulem}
\usetikzlibrary{trees}

% Command for inserting a todo item
%http://midtiby.blogspot.ie/2007/09/todo-notes-in-latex.html
\newcommand{\todo}[1]{%
% Add to todo list
\addcontentsline{tdo}{todo}{\protect{#1}}%
%
\begin{tikzpicture}[remember picture, baseline=-0.75ex]%
\node [coordinate] (inText) {};
\end{tikzpicture}%
%
% Make the margin par
\marginpar{%
\begin{tikzpicture}[remember picture]%
\definecolor{orange}{rgb}{1,0.5,0}
\draw node[draw=black, fill=orange, text width = 3cm] (inNote)
{#1};%
\end{tikzpicture}%
}%
%
\begin{tikzpicture}[remember picture, overlay]%
\draw[draw = orange, thick]
([yshift=-0.2cm] inText)
-| ([xshift=-0.2cm] inNote.west)
-| (inNote.west);
\end{tikzpicture}%
%
}%

\definecolor{light-gray}{gray}{0.95}
% Compensate for fbox sep:
\newcommand\Hi[2][light-gray]{%
  \hspace*{-\fboxsep}%
  \colorbox{#1}{#2}%
  \hspace*{-\fboxsep}%
}

\title{Technical Report}

\begin{document}
\maketitle

\tableofcontents


\section{Introduction}

\section{Background}
Orbiscom Ireland, acquired by Mastercard in 2009, develop a product called InControl. The acquisition allowed Mastercard to incorporate InControl into their Value Added Services, a range of products that tie in closely with their core business, the processing of electronic payments. Until 2010 Mastercard Ireland was still primarily concerned with the development of InControl. A division of Mastercard Labs was setup within Mastercard Ireland in 2011. As Mastercard continues to grow, more and more departments are being given a presence within Ireland. This report documents the activities and projects undertaken while working for InControl. Since their acquisition Orbiscom have become completely integrated into Mastercard and have ceased operating as Orbiscom entirely. InControl has been split into two different products. InControl Direct exists to support Orbiscoms original customers as is managed and maintained entirely within the InControl department. The second product, InControl is an adapted version of the original product. Where InControl Direct is deployed to and hosted by banks, InControl is hosted on BankNet, Mastercards global payments network. In order to explain the services offered by the InControl platform, it is useful to explain how the credit card payment process works.

InControl and Orbiscom are used somewhat interchangeably in the rest of the document. Orbiscom is a keyword used throughout the code base. All of the packages still begin with com.orbiscom \todo{make italic}

\subsection{The Four Party Model}
The credit card payment scheme employed by Mastercard involves four separate parties, for this reason it is referred to as the four party model. There are alternate models in use, but Mastercard employs the four party model as it allows the issuing of payment cards to be handled by separate financial institutions. This leaves less overhead for Mastercard and also insures interoperability between the various financial institutions. The credit card holder typically initiates the payment and is represented by a credit card issuing bank, or an issuer. The merchant involved in the payment is represented by an acquiring bank, or an acquirer. Each successful payment goes through three stages; Authorization, Clearing and Settlement.
\subsubsection{Authorization} The card-holder submits their payment card details to the merchant. The merchants bank, the acquirer, sends a request to Mastercard to identify the card-holders issuing bank. Once this has been determined and the card has been verified the payment is forwarded, by Mastercard, to the issuing bank. It is worth noting that this is the stage in the process where the InControl service is applied if needed. The card-holders bank approves the purchase and blocks the funds in the card-holders account. No money has been transfered from the card-holders account, the money has merely become unusable by the card-holder. The issuing bank forwards the approval to Mastercard, who forwards it to the acquirer who in turn forwards it to the merchant. The payment has now been approved. 
\subsubsection{Clearing} Sometime after the payment has been authorized, usually at the end of the week, the merchant submits all of their authorized payments to clearing. This is a batch process that occurs at certain set times. After a payment has been cleared the funds have effectively been transfered.
\subsubsection{Settlement} Come back too, concerns the actual transfer of funds.

To facilitate this, a fee is applied to each transfer. Interchange is typically charged by the issuer to the acquirer. This is also where Mastercard makes money from each payment. The rate of interchange varies from bank to bank and can even be different depending on the nature of the purchase. In order to sustain credit card usage and adoption the services offered by Mastercard must outweigh this additional fee. This report will make no attempts to the discuss politics surrounding this fee and will simply assume the following. \todo{REMOVE: sometimes I love academia} The interchange fee is to some extent ultimately payed by the merchant. This means that they lose an amount on every purchase made with a payment card as opposed to cash. In order for merchants to continue to accept card payments there must be sufficient consumer pressure for acceptance by the merchants. This is done by incentivising consumers with additional benefits not available through cash payment. Some of these incentives are part of Mastercards core network and come as benefits of using an electronic system such as added security, access to credit and accountability. Mastercards Value Added Services range is a set of products designed to add additional benefits for consumers.
\cite{Something about 4 party, also check I have it the right way around http://tinyurl.com/7bbdq4
 }

\subsection{InControl}
The InControl platform allows card-holders to create virtual payment cards. The virtual card numbers (Henceforth called VCNs) are linked back to the original real card and account and can be used exactly like a normal payment card. At the core of the system is the algorithm used to create VCNs, however similar systems have also existed. The unique feature offered by InControl is that no changes to existing infrastructure are required. If a payment requires InControl to continue processing then the payment is routed to the InControl platform. This happens entirely within the payment network. Neither the merchant nor the acquiring bank has any knowledge that InControl has been applied, so no action is needed on their part. Other similar technologies required the merchant to add knowledge of the process to their point of sale, ie, they had to replace all of their card readers.

Additionally, InControl allows an extensive range of controls to be set on payment cards. Rules can be placed on the amount spent, where it is spent, what it can be spent on, along with many others, can all be controlled. These controls can be applied by individuals to single payment cards, or by banks to ranges of issued cards. The ability to control a persons spending has proved a popular feature. The bulk of InControls traffic is made up of corporate accounts. Many business have found that VCNs are an effective way of controlling employee spending. 

The original InControl Software platform was designed as a generic server framework to host various payments services such as InControl VCNs as well as other third party technologies such as Verified by Visa. The VCN service was developed by Orbiscom. If a bank chose to provide this service they could use the InControl platform to host other financial services as well. This fucntionaThe entire platform is written in Java and configured using XML. Work began around the same time as the first versions of Java Enterprise were being released. This had a very clear effect on the development of the platform. InControl was designed as a generic server framework because no suitable server framework existed at the time.

MasterCard operates Banknet, a global telecommunications network linking all MasterCard card issuers, acquirers and data processing centers into a single financial network. The operations hub is located in St. Louis, Missouri. Banknet uses the ISO 8583 protocol. The network is peer-to-peer mesh network with a set of endpoints. At no time during my internship did I have any involvement with any aspect of Banknet, but it does have one very notable effect on development within the company, the release cycles. Banknet is the core of Mastercards business. One of the most important aspects of an electronic payments service is reliability. If Mastercard is unable to serve it's customers in anyway it would mean a huge blow for their brand, a brand they have invested a substantial amount of money in. Banknet must be reliable. This is achieved in part with an extremely conservative attitude towards updating the software that controls the network. The servers can only be restarted twice a year. This takes a considerable amount of effort. Each node must be updated and "fliped" across the entire network. The network cannot be taken down completely during this time and their are many possible issues that are taken into account, hence the twice yearly cycle. There are a further two periods each year where the network can be updated without restarting the core services. All of this means Mastercard uses a quarterly release cycle, with fixed deadlines. Due to the inflexibility of the release cycle, great care is given to selecting new functionality to be included in each quarters release. These factors, a rigid deadline, fixed requirements and a very strong reliance of reliability in the end product are traits commonly associated with the waterfall design process, and this is the design process employed by Mastercard.

The waterfall design methodology arguably fits well with the requirements surrounding Mastercards core network. Other services, such as InControl are not subject to the same requirements but are non the less hampered \todo{REMOVE} by the employment of waterfall. During my time at Mastercard a switch an agile design process was beginning to get underway within InControl.\cite{I had a source here but it seems it was lost}

\section{Existing System}

The InControl platform is very complex. It was not developed as a single service but rather as a generic server framework to host various payment services. The platform is component based based and hosts a number of generic services and product specific services. The role of InControl has changed over time, with the emphasis now placed the VCN service as Mastercard have no interest in hosting their competitors products.

\subsection{Components Overview}
The following is a brief overview of the components within the InControl system. At present, the system consists of 21 separate components containing about 768,732 lines of code between them.\cite{SLOC} This is just the core InControl product, there are many other auxiliary projects in the subversion repository. A detailed look at each components is carried out in subsequent sections.

The InControl platform is comprised of seven major types of component.
\subsubsection{Client applications}
Client applications include any client interacting with an InControl server. These include the InControl Virtual Card Client and other third part clients, such as Verified by Visa clients. There are two versions of the Virtual Card Application (originally called the O-Card application), the thin and slim client applications both communicate all requests to the VCNs server via the InControl web server environment.
\subsubsection{Web Server Environment}
The web server environment acts as a gateway to the InControl Dynamic servers for the external client applications. The InControl Payment platform typically requires that a web-server is available. The web server environment may include web servers or application servers, or a mix of both. The main functions of the web server environment is as follows:
\begin{itemize}
\item Provide the content files for InControl client applications, e.g. configuration files, web assets. The client content can be hosted on any plain old web-server.
\item Host the InControl servlets. The main function of the servlets to to preform message format conversion to enable InControl servers to process messages from the client applications and the reverse. InControl Direct uses \todo{Athena, wrong usage}, Mastercard InControl uses \todo{GET NAME} as each operate within different networks. This is a clear benefit of a modular design, only one component had to be switched out then the platform was moved to a different network \todo{LOL NO}. An application server providing the appropriate servlet environment is required. The \todo{Athena, wrong usage} supports a number of servers including WebSphere and Tomcat, allowing it to be easily integrated with existing infrastructure. If there is no existing infrastructure it can be run inside the InControl Apollo framework. 
\item Host InControl sessions and authentication manager. The session and authentication manager allows for integration with existing an customer authentication API. \todo{Athena, wrong usage} can also provide servlets for session management and authentication.
\end{itemize}
\subsubsection{InControl Servers}
The InControl servers provide the processing core for the InControl payment platform. A generic server framework is provided for hosting the generic servers, e.g. maintenance and the payment products services, e.g. Virtual Card Number services. The generic InControl server architecture is based on a dynamic server framework. The InControl servers reside on Banknet, while the InControl Direct servers typically reside on a customers, e.g. An Bank, internal network. The services provided by the platform can include Online Services, Batch Services or Scheduled Services.
\begin{itemize}
\item Online Services: The platform can be used to provide a number of online services including registration, session and authentication, authorization services and client support services.
\item Batch Services; Batch services read a file as input and apply changes to the server data based on the contents of each record. They can also be run as command line utilities. The platform provides card settlement, registration and maintenance as batch services.
\item Scheduled Services: These are programs that can be configured to run periodically in order to preform housekeeping tasks. They can be configured to run inside one of the installed InControl services or to run as standalone processes.
\end{itemize}
\subsubsection{InControl Database} At the core of the system is a relational database shared by all components. The system uses an Oracle database because \todo{FIND OUT}. The database provides both data storage and configuration information to every component within the system. \todo{YOU COULD PROBABLY EXPAND THIS A LOT}
\subsubsection{Customer Service System} The customer service system is web application made available to the clients customer service representatives for dealing with customer service issues in relation to the InControl Platform. The Customer Service System \todo{What did marketing call it this week} consists of two components. A web server used by the client and a backend server used to query the InControl systems database.
\subsubsection{APIs} APIs are available to either access functionality of the system, ex. the registration API, or to provide functionality to the system, ex. the user authentication and session management for the platform.
\subsubsection{Administration Programs} The InControl Platform comes with a number of administration programs that are used to preform or initiate housekeeping tasks. These programs can be run from the command line. Examples of such programs are the archiving process and the Server Status Controller.

\subsection{Web Server Environment} \todo{This is very likely all wrong, come back to later}
The web server environment provides the link between the external client applications and the InControl Servers, the core of the InControl Platform. Typically web servers and, optionally, application servers are used to provide the appropriate web server environment.

The web server environment typically hosts the Orbiscom servlets. The Session and Authentication Manager can also be hosted within this environment but is typically looked after by the customer in the case of InControl Direct, or by Mastercard in the case of InControl.

\subsubsection{Content Hosting}
A web server is required to provide the content files fro Orbiscom client applications. These files are served dynamically each time the client connects to the URL. Any plain old webserver can be used \todo{Possibly BS}

\subsubsection{InControl Servlets} 
The Web Server / Application Server typically hosts the InControl servlets. Athena is the servlet system used to support the InControl Platform \todo{THIS IS WRONG}



\subsection{InControl Server}
This section describes the InControl server architecture, incorporating the generic server framework. The InControl server architecture is described in the following areas:
\begin{itemize}
\item Generic Server Framework
\item InControl Services Overview
\item Configuration and Installation
\item Communications Security
\item Operational Requirements
\end{itemize}

\subsubsection{Generic Server Framework}
The Generic Server Framework is based on a dynamic server framework. Apollo and Atlas provide the generic server framework.
\begin{itemize}
\item Apollo provides the generic, dynamic, extensible framework for clients and servers.
\item Atlas provides the InControl HTTP and XML dispatchers.
\end{itemize}
These two components are at the core of the InControl platform. Everything is built on top of them. Together they create a full generic server framework with capacity to manage authentication, thread synchronization, component configuration, database connection and connector pooling, security and session management. As they have such an important role, the deserver a detailed analysis. \todo{INSERT DETAILED SOURCE LEVEL ANALYSIS}

The InControl Servers are UNIX based and implemented entirely in Java. They are designed to be platform independent and will run on any platform that supports an appropriate release of the JVM, at present InControl targets JVM 1.6.
\todo{REDO DIAGRAM}
InControl servers can be configured to listen on one or more communications interfaces for incoming connections. Each listener can be configured to process one communications protocol. The platform supports HTTPS, HTTP, raw TCP/IP, MQ Series and X.25.

Each listener can be configured to pass on requests to different types of content handlers. The most common form of content handler is the InControl XML content handler, which process messages that are in a format specified in the InControl Message Document. \todo{GET THAT DOC. Also, I presume this format is now the mastercard format, is it translated into Orbiscom format or what?}

Other message handlers can process various authorization card scheme formats, for example Visa and Europay handlers.

InControl Servers can be configured to dynamically update various configuration items upon receipt of a signal from a Platform Administration Utility.

The following is a list describing the behavior of a InControl server.
\begin{itemize}
\item Connection: At boot-up a number of dynamic servers are created to process jobs. Each Server creates a number of connectors. These connector classes allow the server to either listen for requests or connect to external services. The classes, which implement the actual connector protocol, e.g. TCP/IP, are specified in the Dynamic Server configuration. This allows the connectors to be swapped out easily, The connector specification contains the name of the implementing class and a port number. There may also be a number of optional parameters defined depending on the protocol. As the Connectors are implementations of a Connector interface, there are no mandatory parameter. As such, all of the parameters are in fact optional, but omissions will result in exceptions being thrown. These are caught and a default value is returned if specified. This would appear to be a bad way of returning a default value as exception handling is being used for control flow. \todo{Get input on this}
\begin{verbatim}
    <Connector
        Class="com.orbiscom.apollo.net.TCPServerConnectorFactory"
        Port="12400"
        TcpNoDelay="true" 
        ReadTimeout="3"/>
\end{verbatim}
\item Transport: A protocol handler specifies the Network Transport Protocol that is implemented on the connectors, e.g. HTTP. The protocol handler can specify many content handlers. The handler specification contains the name of the implementing class and a number of handlers. The handlers are indexed by a request type. In the case of HTTP the request type is the path. The handler specification body can be as simple as just the name of the implementing class or much more complex. An example is provided below. The first handler specified is an XML Dispatcher. This content handler converts the incoming HTTP content to an InControl XML request for processing. The other elements of note are the implementing class and the HandlerSet. The handler set contains links to more configuration files. These files contain the mapping from requests to the handlers to service them. Different versions of the InControl protocol may require different handler sets. Great care has been taken to ensure that each new release is backward compatible with the previous version, but if needed separate handlers can be used.
\todo{FORMAT THIS CORRECTLY, need a proper, line wrapping way of doing code}
\begin{verbatim}
<ProtocolHandler
		Class="com.orbiscom.apollo.net.HTTPProtocolHandlerFactory"
		MonitorHandlerTimes="true">

		<Handler
			RequestType="/pulse">

			<Task
				Class="com.orbiscom.atlas.xml.Dispatcher"
				RequireAuthentication="true"
				LogAuthenticationErrors="true"
				UseEncryption="true" >

				<DocumentBuilderFactory>
					<Property Name="NamespaceAware" Value="true" />
				</DocumentBuilderFactory>

				<OilContext
					MessageTimeZone="UTC">
				</OilContext>

				<MessageContext>
					<Variable Name="DataExtract.Source" Value="RetailXMLAPI" />
				</MessageContext>

			<HandlerSet>
				<VersionedHandlerSet Version="12.4" xlink:href = "metahandlers" />
				<VersionedHandlerSet Version="12.4" xlink:href = "pulsehandlers" />
				<VersionedHandlerSet Version="12.3" xlink:href = "pulse12q3handlers" />
				<VersionedHandlerSet Version="12.4" xlink:href = "flexhandlers" />
			    <VersionedHandlerSet Version="12.4" xlink:href = "ipchandlers" />
				<VersionedHandlerSet Version="12.4" xlink:href = "evcnhandlers" />
                <VersionedHandlerSet Version="12.4" xlink:href = "sbchandlers" />
            </HandlerSet>

        </Task>
    </Handler>

    <Handler
        RequestType="/schema">
        <Task
            Class="com.orbiscom.atlas.util.SchemaServlet">
        </Task>
    </Handler>

</ProtocolHandler>
\end{verbatim}
An entry in a handler set follows. The example provided is a particularly complected specification to illustrate the widest possible range of functionality. A simple hander consists of a request type name and handler class. In the example entry a constraint and various preprocessors are specified. The constraint classes are built into the server framework and check certain conditions are met before further processing is done. In this case, the supplied PAN (Personal account number or the number embossed on a credit card) is checked to make sure it is the correct length. The preprocessor takes the supplied PAN and maps it to a unique ID, which is then used to index the account in subsistent operations. Each preprocessor falls through to the next so the order is important.

\todo{formatting again}
\begin{verbatim}

    <Handler
        RequestType="GetVCNListRequest">
        <Task
            Class="com.orbiscom.pulse.oil.GetVCNListXMLHandler">

            <Constraints>
                <Attribute Name="Pan" Constraint="PanConstraint" />
            </Constraints>

            <OilProcessors>
                <OilProcessor Class="com.orbiscom.pulse.oil.mapper.PanMapperProcessor" />
                <OilProcessor Class="com.orbiscom.pulse.oil.obo.CheckThirdPartyAccessProcessor" />
                <OilProcessor Class="com.orbiscom.pulse.oil.GetVCNListHandler" />
            </OilProcessors>
        </Task>

        <Task
            Class="com.orbiscom.atlas.mastercard.audit.AuditLogger"
            EventType="com.mastercard.common.jal.events.EventType.ACCESS_CARDHOLDER_DATA"
            AuditMessage="Get VCN List" >
        </Task>
        
        <ErrorTask
            Class="com.orbiscom.atlas.mastercard.audit.AuditLogger"
            EventType="com.mastercard.common.jal.events.EventType.ACCESS_CARDHOLDER_DATA"
            AuditMessage="A request to Get VCN List failed" >
        </ErrorTask>
    </Handler>

\end{verbatim}
\item Application: The XML Dispatcher determines the issuer \todo{Clarify the term Issuer at the start} the request is intended for and starts a transaction in the database configured for that issuer. The XML Dispatcher iterates through all the elements in the request message. For each node, the dispatcher determines the relevant application handler and runs it. The handlers are run in the order specified by the request. The response XML document is created and each application handler appends its response to this document.
The platform is state-full. The dispatcher maintains the database transactions per request by creating a massage context. This context object object is passed to each application handler in turn. The dispatcher only commits the changes when every handler working on the request returns successfully.
If an exception occurs while iterating through the nodes the dispatcher roles back the database transactions. Further processing stops and failure message is returned. The response will specify a return code and a message to indicate the nature of the failures. At present, the core system, e.g. Atlas and Apollo, have X \todo{get that} individual error codes defined. Each component may also define it's own failure codes, for example, the Retail API Server has Y \todo{get that}. These codes are of the format NNXXX, where NN is the component name and XXX the error code. The code is always followed by a textual description of the failure. ex, AT001 General Error. Here AT is the component name, Atlas \todo{Atlas or apollo?} and 001 is the error code. Often it is fairly trivial to quickly trace the cause of an error because of this. However, error codes are generally defined for well understood and anticipated error. /todo{Get confirmation on this, maybe expand?}.
An additional task is also run in this instance. The event is written to an audit file for logging purposes. The audit file is a separate database supplied by Mastercard called the data-warehouse. The idea is to offload logging from the main system \todo{Confirm Mastercard bit}.
After the request has been processed the XML dispatcher completes the audit message by writing a trailer recored. This record indicates whether or not the transaction was successful.
I was curious about the reasoning behind state-full request processing. Once again this can be attributed to the difference between taught best practices and real world scale. The InControl platform, as evidenced, is not trivial. Many of its operations and use cases are complex and it makes sense to preserve state in many instances. This keeps the size of any one operation to a manageable size and allows reuse. While changes in a specific request may (and have) had a negative effect on other requests it is still a much better approach than trying to make every request stateless, and thus huge. \todo{More of this questioning and "critical thought", Also make sure if have drawn attention to things being different in the enterprise by now in the text. Also that last paragraph is particularly shite}
\item Client Mode: These are servers than can be configured to initiate connection to another server rather than listen on a port. The poll this connection for requests. They can use same protocol and same content handlers as any other InControl Server. An example would be the InControl Registration server when communicating over MQ Series.
\end{itemize}
\subsubsection{Configuration}
All communication from applications external to the server is configurable. InControl platform components are highly configurable and the configuration files have grown substantially, accounting for just over 7\% of the code base. Every component has it's own set of configuration files. There is also a set of common configuration files\todo{WHATS IN THE COMMON}. The system consists of five general types of configuration file. Each file is used to configure a different aspect of the system. Any file may link to several other files, but the following can be viewed as the five root configuration files.
\begin{itemize}
\item Top Level configuration: This is the first configuration file read on startup. It is passed as a command line argument and is responsible form specifying the various configuration items the server has access to. The following configuration files are contained as nodes.
\item Logging: The log4j configuration file contains the logging configuration for the system. Log4j is a third party component used by the InControl platform. It employees configurable logging levels for different aspect of the system. The current debug level can be specified in the file. During development full logging will be enabled, but in a production environment INFO or ERROR level are used. This ensures only relevant information is recorded. It also keeps writes to log files down. IO operations are generally very costly, so being able to easily change the logging level is very beneficial.  Additionally, a typical log4j configuration file will contain a number of loggers. These are specific to classes or packages in the classpath of the running Java application. Loggers contain a logging level at a minimum.
\begin{verbatim}
<logger name = "com.orbiscom.atlas.range">
    <level value = "info"/>
</logger>
\end{verbatim}
Appenders are also specified. Crucially, these elements define where a log is written and the implementing class, as well as information such as the character encoding used and time-stamp patterns.
\begin{verbatim}
<appender name = "Debug" class = "org.apache.log4j.DailyRollingFileAppender">
    <param name = "File" value = "log/debug.log"/>
    <param name = "Append" value = "false"/>
    <param name = "Encoding" value = "UTF-8" />
    <layout class = "org.apache.log4j.PatternLayout">
        <param name = "ConversionPattern" value = "%d{dd,MM HH:mm:ss:SSS} [%t] %-5p %c{2} - %m\n"/>
    </layout>
</appender>
\end{verbatim}
\item Database configuration: The database configuration specifies all the databases that an InControl Server accesses. It contains the database URL, the username, password and the database driver class. This information resides in a ConnectionPool node.
\item Dynamic Server configuration: The Dynamic Server configuration section defines the operation of the server, the actions it will perform, supported message types and the supported protocols. The ProtocalHandler element shown above is from a dynamic server configuration file. It also contains links to the handlers used by the servers, again illustrated above.
\item Classloader configuration: \todo{Relevant? If not, change 5 to 4 above}
\end{itemize}
\subsubsection{Communications security}
Inter communications between the components and also between a component and an external server may be secured. SSL can be used, in such cases certificates must also be configured. Signed data an also be used, in which all communication is signed with a preshared key. The Java Cryptography Extension framework is used to preform data signing. JCE was used as it is standard cryptography framework for the Java SDK. The system signes the data with hash-based message authentication codes using SHA-1. The keys are contained within jar files as this is the easiest and safest way to store them.

\subsection{Session \& Authentication Service}
The Session \& authentication Service is responsible for verifying the cardholders credentials, thereby ensuring that they are valid users of the InControl system.
A flexible authentication service is provided, where Issuers have the option of implementing any of the following authentication sessions.

\subsubsection{InControl Internal Authentication}
InControl is capable of managing complete authentication. The user credentials are stored in the inside the InControl database. SHA-1 is used to encrypt the passwords.
Verification rules for User credentials can be configured in the database. These rules are validated when creating the user credentials, e.g. Minimum number of numeric characters required, max invalid logins etc. The number of possible rules has grown considerably, from three or four to over thirty five. These values for these requirements have stayed fairly static for a number of years. A default set are usually set for each new issuer, and changes made thereafter if required. 

\subsubsection{External Authentication}
The InControl Session \& Authentication Manager (SAM) may be hosted in the web server environment or on the InControl Server platform. The SAM acts as a wrapper for an issuers authentication and sessions APIs. InControl servlets authenticate VCN requests through the SMA API rather than linking directly through the issuers API. This is again part of the modular design employed throughout the project. The various servlets do not require customization for each customer, only the customer specific SAM. This modularity is only made use of in InControl Direct. After the InControl platform had been updated to operate with Banknet, session and authentication were no longer an issue as Banknet takes care of them.
The SAM classes are specified in a servers properties file.

\subsubsection{Session Management}
To avoid the requirement for the client to send their complete user credentials in every message the InControl platform supports sessions management. When a clients credentials are successfully validated a unique 40 digit session ID is generated using a cryptographically secure hash algorithm and sent to the client. This token and the unique user ID are stored in the InControl database in a temporary table. On future requests the client need only send their token and the platform will accept the message. The timeout for a session can be configured in the database.

\subsection{Client Support Services}
\todo{later}

\subsection{Online Registration \& Maintenance Service}
The Registration and Maintenance of a cardholder and card information in the InControl database is required to allow cardholders to avail of the payment services provided by the InControl Platform. This service is provided by the Retail API Server and the implementing component is Pulse. The example config files above are taken from the Retail API Server.

\subsubsection{Functionality}
The Retail API Server receives and processes XML formatted requests. The server listens for incoming valid XML requests from an external system on a configurable port. A defined attribute in the XML request will determine the action to be taken. The service can also supply default values for any non-specified data. These values are specified in the implementation and cannot be configured.
Once the request has been processed successfully, a response will be sent back to the source informing them of the success or failure of the request.
The Retail API server encompasses the majority of the InControl Platform functionality and is used by several other components, in addition to being one of the customer hooks into the InControl Platform.
\subsubsection{Message Formats}
All messages sent to and processed by the Retail API Service are XML documents transported by HTTP requests. The HTTP requests are fairly ordinary, containing information such as host, port, service etc.

The HTTP message body contains the information to be processed by the Retail API Server. The root node is known as the requester. A requester can contain N individual requests. If any one request fails, the overall requester fails. The response is structured similarly, the response root node contains the response value, e.g Success or Failure, and N response elements containing the response data specific to each request.

An example XML request is shown, populated with mock data.
\begin{verbatim}
[<?xml version="1.0" encoding="UTF-8"?>
<OrbiscomRequest IssuerId="1" Version="12.4">
  <AddRcnBinRangeRequest IssuerId="99993">
    <Range CardType="83" CpnType="DF" EndRange="1234569999999999" LanguageId="1" Prefix="123456" StartRange="1234560000000000" Status="A"/>
  </AddRcnBinRangeRequest>
  <SetBinRangePropertyRequest IssuerId="99993">
    <Property Name="ICA" Value="009661"/>
  </SetBinRangePropertyRequest>
  <SetBinRangePropertyRequest IssuerId="99993">
    <Property Name="AICABRC" Value="5"/>
  </SetBinRangePropertyRequest>
  <SetBinRangePropertyRequest IssuerId="99993">
    <Property Name="IPCAMSRequired" Value="0"/>
  </SetBinRangePropertyRequest>
  <SetBinRangePropertyRequest IssuerId="99993">
    <Property Name="DefaultIPCProfile" Value="DEFAULT"/>
  </SetBinRangePropertyRequest>
  <SetBinRangePropertyRequest IssuerId="99993">
    <Property Name="IPCPorductType" Value="FC"/>
  </SetBinRangePropertyRequest>
</OrbiscomRequest>
\end{verbatim}


\subsection{Batch Registration and Maintenance Service}

\subsection{Rules \& Controls}

\subsection{Virtual Card Generation}

\subsection{Virtual Card Settlement}

\subsection{Virtual Card Authorization}

\subsection{OIL}
Orbiscom Interface Language is an in house Interface Definition Language. It is used to specify how various components communicate with each other. It can be used to specify basic data objects, request definitions and their behavior. It is a fully fledged language with its own compiler. The compiler generates program code \todo{How is not just a code generator?} that can then be compiled. At the moment, Java and Flex \todo{Correct?} can be generated using OIL.
OIL is used throughout the InControl Platform. Virtualy every component uses it in some respect. Even the components that do not use OIL directly, e.g. do not call the OIL generated code, use OIL indirectly as all of the XML messages are generated by OIL classes.
During my time at MasterCard I too used OIL extensively. If a new request is needed it must be created using OIL. There are several steps involved, beginning with creating the database code to service the request. The stored database procedure is invoked by and OIL request handler. This handler must be written manually. I wrote a script to automate this process somewhat. The handler is invoked by an OIL generated handler, which is configured to run when a specific request comes in. This request is again an OIL generated type. The steps to add a new request to a component called Cronos follows
\subsubsection{Create the stored procedure}
A stored procedure is set of pre-compiled SQL statements stored inside a database. This example is just a simple select statement but any SQL is valid. Stored procedures are saved to \textit{release}/database/procedures/src/o\_\textit{module}.pls, e.g. \textbf{\textit{r12q4/database/procedures/src/o\_cronos.pls}}. \todo{Does that look a bit much}Each stored procedure package consists of two sections, procedure prototype definitions and procedure bodies. This is the same as any programming language, the prototypes are function heads and the implementing code is contained within the body.
A definition:
\begin{verbatim}
PROCEDURE GetIssuerConfig (
        p_issuer_id IN VARCHAR2,
        r_issuer_config OUT Globals.ref_cursor);
\end{verbatim}
Here the procedure takes a string and a reference cursor. A cursor acts like a pointer to the result set returned by the query. A ref cursor is similar, the key difference being that it is not bound to a single result set and can also be passed back up to the client, which is somewhat important. If no result set is returned, ie, an insert, no ref cursor is need.
The implementing code. This is just standard SQL code
\begin{verbatim}
PROCEDURE GetIssuerConfig (
        p_issuer_id IN VARCHAR2,
        r_issuer_config OUT Globals.ref_cursor)
    IS
    BEGIN
        OPEN r_issuer_config FOR
            SELECT issuer_name, authentication_type, authorisation_type, session_timeout,
            authenticate_pan_password, allow_change_email_addr, allow_set_user_name, create_date, update_date
            FROM issuer_config
            WHERE issuer_id = p_issuer_id;
    END;
\end{verbatim}
The stored procedure must then be loaded into the database. If there are any errors in the stored procedure file loading will fail and the entire file and all of the contained procedure will be rejected. It is generally a good idea to test the procedure works as intended before writing code to access it. It can be executed by specifying the package and procedure name. If the procedure requires a reference cursor it must also be declared
\begin{verbatim}
var blah refcursor
execute 'cronos12q4db.GetIssuerConfig('1', :blah);
\end{verbatim}

\subsubsection{OIL Interface}
There are two files used to define the classes OIL generates, the OIL definition file and the OIL procedure definition file. The procedure definition file is nearly identical to the stored procedure definition file, so it may be easier to write it first and then the definition file.

The OIL definition file, e.g. \textbf{\textit{cronos.oil}} is used to define the request and any data types needed by that request. The file begins with an interface definition, this is the package where the generated code is stored. This is optionally followed by any imports needed. This allows OIL files to be split up very easily. Typically, that is, by convention only, data types are defined. The two built in data types are \textbf{String} and \textbf{Number}. Any user defined data types are collections of these and other used defined data types. Default values are supported. This is where the default values mentioned in the Retail API Server section come from. They are specified in the OIL files and are thus hard coded into the generated Java code.
\begin{verbatim}
 type IssuerConfig
    {
        String IssuerName;
        Number AuthenticationType;
        String AuthorisationType;
        Number SessionTimeout;
        String AuthenticatePanPassword;
        String AllowChangeEmailAddr;
        String AllowSetUserName = "N";
    }
\end{verbatim}
The request definitions follow the type definitions
\begin{verbatim}
request GetIssuerConfig(
        required String IssuerId)
    {
        response
        {
            IssuerConfig IssuerConfig;
        }
    }
\end{verbatim}
This is a very basic example of what OIL is capable of and will result in a basic data model Object and request definition. OIL supports annotations to support various features, such as comments to be included in the generated code and enforcing mandatory members in request parameters etc. Arrays are also supported. They can be restricted to containing elements of a specified type or multiple elements of different specified types. OIL definition files also have access to a Context Object. Values can be stored and retrieved from this Object when the generated code is running. This is particularly useful for large and complex operations. Values can be saved in the context to avoid having to pass them between requests.

The OIL procedure file maps the values returned by the stored procedure to the members of the OIL Object returned in the query response. It is an Object relational mapping similar to technologies like Hibernate \todo{Might be stretching it a bit there}
\begin{verbatim}
 procedure GetIssuerConfig[Cronos12q4DB.GetIssuerConfig] (
        String IssuerId => p_issuer_id)
    {
        output
        {
            IssuerConfig IssuerConfig <= r_issuer_config
            {
                String IssuerName <= issuer_name;
                Number AuthenticationType <= authentication_type;
                String AuthorisationType <= authorisation_type;
                Number SessionTimeout <= session_timeout;
                String AuthenticatePanPassword <= authenticate_pan_password;
                String AllowChangeEmailAddr <= allow_change_email_addr;
                String AllowSetUserName <= allow_set_user_name;
            }
        }
    } 
\end{verbatim}

\cite{Orbiscom Payments Platform Technical Architecture}
\cite{http://wiki.orbiscom.com/index.phpPlatform_Documentation}
\cite{Cite more things so it doesn't looked as much like pasta, and spread them thought the text}
\cite{Also cite source code all over the place}
\cite{Stored procedure for dummys}


\section{Development Process}

\section{SMART}

\section{Conclusion}

\section{References}

\section{Appendices}


\end{document}
