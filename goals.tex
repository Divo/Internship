
\documentclass[a4paper,12pt, titlepage]{article}
%\usepackage{color}
%\definecolor{light-gray}{gray}{0.95}

\usepackage{xcolor}
\usepackage{alltt}
\usepackage{url}
\definecolor{light-gray}{gray}{0.95}
% Compensate for fbox sep:
\newcommand\Hi[2][light-gray]{%
  \hspace*{-\fboxsep}%
  \colorbox{#1}{#2}%
  \hspace*{-\fboxsep}%
}

		% This is my own macro !!!

\title{Internship Goals}


	
\begin{document}
\maketitle						% automatic title!
\begin{center}
  {\large Steven Diviney} \\
  08462267 \\
\end{center}
\section{Overview}
Mastercard Ireland are expanding quite rapidly, but as of writing the two main departments are InControl and Labs. I am working as part of the InControl team. Mastercard acquired the proprietary InControl platform to gain a competitive advantage. The InControl product allows a Mastercard card holder to create virtual credit card numbers off their real credit card number. Controls on transaction amount, number of usages, types of purchases, can be set on these virtual cards. These virtual cards numbers can be used in exactly the same way as a real credit card number. This was the main advantage InControl had over similar competing technologies, no changes were needed to any existing  MasterCard Banknet infrastructure bar a service to route virtual credit card numbers to the InControl platform once the real card number is identified as using the InControl service. My role as an intern is to expand and develop the various administration and testing tools used by the team and their customers. My goals for the internship are to make a positive contribution to a real world development team, to understand the processes employed in professional software development and to expand and refine as many skills as possible. 
\section{Goals}
\subsection{Investigate existing system}
Mastercard Ireland have developed a platform called InControl. The platform is quite complex, consisting of 21 separate components. The platform provides five services, Purchase Control, Family Solution, Small Business Control, Issuer Portfolio Control and Account Level Services, and the Retail API which allows third parties to interface with the platform. The purpose of each component and how it interacts with the other needs investigation. 
\\The intended outcome of this goal is to understand the system as a whole and to understand how each component contributes to the system and the products offered to the end user. 
\\In order to assess accomplishment of this goal I will have to engage with the various developers responsible for creating and or maintaining each component. 
\\Possible challenges are the lack of existing documentation to this end. Also, as the platform has evolved over the last 10 years, various developer have left the team.

\subsection{Comprehend technologies employed}
InControl uses a wide variety of technologies. Many of these were created in house but third party technologies are also used.
\\The intended outcome is find out the reasons for the various technical decisions made throughout the lifetime of the project.
\\This goal is probably best done in conjunction with the first. Every time a new component is investigated, the technologies employed should be scrutinized. This should result in a list of reasons that can be compared to list of technologies gathered with the first goal.
\\Again, possible challenges include poor documentation and various responsible individuals not longer being available. Conflict of interest may also play an interesting role, as some people in the team may not have been happy with certain decisions. 

\subsection{Reason behind infrastructure}
The reasons for the technologies employed by InControl and the reasons for the design of the various components should also be found.
\\Again, this should be done in tandem with the first goal, resulting in two lists that can simply be compared. If a component lacks justification for it's design then this goal has not been completed.

\subsection{Development Process}
Documenting the development process, or software development life cycle, employed by the InControl team would be of significant value to myself and perhaps to the team. The reasons for the infrastructure behind the supporting technologies used by the team, as opposed to the infrastructure of the end product as above, will also be investigated.
\\The intended outcome is detail how any particular component is designed. This may vary for each component and developer but at a broad level the steps should include:
\begin{itemize}
\item Requirements gathering and analysis
\item Design
\item Testing
\item Deployment
\item Managing and maintenance
\end{itemize}
The best way to assess this goal is to talk to the developers and see if they agree with my conclusions.
\\Possible challenges include variance in method across components and developers. The InControl team use quite a free flowing development process (The reason for this would be interesting, particularly if it was a conscious decision) and it may be difficult to pin down.

\subsection{Development of SMART}
I have been assigned to maintain and extend SMART, an in house tool used for testing and administering various functionality of the InControl system.
\\The objective here is to detail the process I used in developing SMART.
\\This will be done with respect to the following:
\begin{itemize}
\item Objectives
\item Process used
\item Problems and challenges
\item Evaluation and testing
\item Design 
\end{itemize}

The considerations taken when developing a system in a commercial environment should also be discussed as they are quite significant.

\subsection{Evaluation of SMART}
Evaluation of SMART with respect to the InControl system.
\\The goal here is to see how SMART fits in with the rest of the InControl platform.
\\In order to assess this I should consult with other developers to see if they agree with my conclusion. SMART is also used extensively by the QA department so input should also be sought from members of that team.
\\There may be some difficulty in assessing the completion of this goal. By the end of the internship I will (hopefully) be the person most familiar with SMART.

\subsection{Development of Spend Control and Purchase Control Client Support Services}
After the work on SMART has been completed I will begin work on the web client support services. The purpose of this system is to give InControl customers, in this case credit card issuing banks, a window into the InControl platform. The work I will be doing is functionally similar to SMART, but as the CSS is a customer facing product it must be done to a much higher standard.
\\The objective and evaluation is similar to the development of SMART. As this is a customer facing product a contrast of the methodologies employed vs those employed in the development of SMART would be very beneficial. In particular, I would like to identify what is needed to bring a product to a release ready state.
Again, this will be done with respect to the following:
\begin{itemize}
\item Objectives
\item Process used
\item Problems and challenges
\item Evaluation and testing
\item Design 
\end{itemize}
The biggest potential problem is a lack of skill on my part to ensure what I create is of a sufficient standard.

\end{document}             % End of document.
